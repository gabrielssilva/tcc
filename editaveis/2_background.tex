\chapter{Background}
\label{chapter:2}

The absence of a standard protocol for the federation of social networks
hinders the development of interoperable applications, as it leads to
the adoption of divergent technologies. This segmentation affects the
network effect and does not help on the emergence of a \textit{de facto}
standard.

To discuss the standardization of federated systems, it is interesting not only
to analyze the community effort to discuss and adopt specifications, but also
to identify the reasons that make it difficult to reach a consensus. The
subject is incipient in the literature, so we had to resort to reviewing
existing community discussion in the relevant mailing lists. Given the nature
of the World Wide Web Consortium (W3C) as the standards body of the Internet,
we used the archives of the W3C's Federated Social Web Community
Group\footnote{\url{http://w3.org/community/fedsocweb}}.

In the next subsections, we present a documentation of the existing initiatives
and attempts to establish protocols for the federation of social networks,
describing the used technologies and the state of pratice in different
projects.


\section{OStatus}

OStatus\footnote{\url{http://w3.org/community/ostatus}} is a protocol suite to
enable real-time interaction between social networks.  It was proposed by Evan
Prodromou for the implementation of StatusNet, a federated microblogging
network that later originated the GNU
Social\footnote{\url{http://gnu.io/social}} project.

The project obtained visibility in the community, and had its W3C community
group\footnote{\url{http://w3.org/community/ostatus/}} created in 2012, one of
the few formal attempts to propose a federation protocol. However, the group
did not produce any results so far. The OStatus specification received a fair
amount of criticism over its limitations, and did not advance into a standard
accepted as good enough for all participating projects.

OStatus is defined as a combination of protocols, each covering a part
of the interactions that make federation possible. It incorporates
standards that already existed when it was proposed, such as Atom/RSS,
used for syndicating content from different sources.  OStatus proposes
to arrange these standards and protocols in a suite for real-time
interactions between servers, automating the flow of social interactions
among users.

The specialized protocols that are included in the OStatus specification
are described below.


\subsection{PubSubHubbub}

PubSubHubbub specifies a distributed system for publishing and
subscribing. It makes use of central hubs used by services to publish
new contents and updates, and by other services to subscribe, receiving
real-time notifications on new activity. It is an extension to
technologies such as Atom/RSS, that depend on users manually fetching
for updates.

\subsection{WebFinger}

WebFinger is a protocol used for discovering information about entities
based on standard identifiers~\cite{rfc7033}. It aims to solve the problem
of sharing identities among servers.

The specification requires every resource to be identified by a URL.
WebFinger servers should respond to HTTP requests, providing information
in JSON or XML formats. The only information required for other servers
to retrieve information about any entity is its resource URL, which can
be generated with public information using a LRDD process~\cite{lrdd2010}.

\subsection{ActivityStreams}

The ActivityStreams specification defines a format to represent activity
in social networks, such as ``Bob started following Alice'', ``Alice
posted a new message'', etc. It aims to standardize entity formats,
making it easier to share and consume those objects from different
servers. The latest specification proposes the use of JSON objects
with a set of predefined attributes, including the action type, involved
users, and content data.

\subsection{Salmon}

Combining solutions like Atom and PubSubHubbub allow publishing and
updating contents across servers in real-time. However, to allow a
content to be updated -- for example, adding a comment to it -- from any
of those servers, it is also important to make sure the content state
will always be the same in the entire network.

Salmon is a message exchange protocol that provides a way to change and
track the state of contents across servers. It proposes a standard
process to securely exchange information among servers, merging
all modifications in a stream of messages.

\subsection{OStatus Current State}

The OStatus project was an important step towards the standardization of
a federation protocol, what is clear given the creation of a W3C work
group.  However, the specification was not complete enough to fulfill
the requirements of most projects, mainly due to its limitation to
public content and the lack of mechanisms to handle privacy.

In 2012, Evan Prodromou announced the development of \textit{pump.io},
another approach for federating social networks, with more modern
technologies. It contributed to the weakening of the OStatus community,
which is not currently active. The StatusNet project is also no longer
active.

Beyond there being no active community to maintain or develop the
project, most of the technologies used in OStatus suite evolved. OStatus
still specifies the use of technologies considered by many as legacy,
such using Atom (instead of JSON) for ActivityStreams messages.

Despite using technologies said to be outdated, OStatus is still used on the
development of federated social networks, such as GNU
Social\footnote{\url{http://gnu.org/s/social/}}, which describes itself
as a continuation of StatusNet, and the Mastodon
project\footnote{\url{https://github.com/Gargron/mastodon}}, which promises
compatibility with GNU Social.

\section{Diaspora}

The Diaspora project\footnote{\url{http://diasporafoundation.org}} proposed the
implementation of decentralized social networks in response to concerns raised
on privacy and freedom in centralized social platforms. Its first version was
released on September, 2010 as a result of a crowdfunding campaign. By August,
2012, a community of users started to maintain it.

Diaspora's main selling point is avoiding content centralization, with a lack of
central control over user data, by building a network of personal servers,
called pods. Each Diaspora pod stores only the data of its own users, and
allows them to interact with users on other servers through a federated
network.

The Diaspora federation protocol was created to converge with OStatus as
soon as the latter supported private contents, what did not happen so
far. Most of the protocol specification relies on the concept of remote
users, and on an exchange mechanism.

\subsection{Remote Users}

A fundamental concept to implement federated networks using the Diaspora
protocol is taking into consideration the existence of remote users.
Most applications only recognize local users, that is, those who have
their credentials and profiles saved in the local database. However, to
enable interaction with users from other networks, at some point it is
necessary to handle users that do not exist locally.

Diaspora classifies its users in two categories: local and remote. While
local users follow the classic implementation, remote users only
interact with the application through federation mechanisms. Having two
category of users may affect the project architecture, and this concern
should ideally be taken into consideration on early design stages.

\subsection{Message Exchange}

In the Diaspora protocol, messages are exchanged using a subset of the
Salmon protocol. Essentially, the Diaspora specification describes how
the message should be built, encrypted, and sent to the Salmon endpoint
of the destination pod.

The payload of a Salmon message is an object that represents an
activity in a pod. Those objects are called entities and can represent
content publications, private messages, comments, retractions, or even
notifications of follow and unfollow actions.

Every message is sent using HTTP to certain endpoints in the destination
server.  The endpoint depends on the nature of the entity. For example,
private messages are sent to user-specific salmon endpoints, while
public contents are sent to public endpoints.

\subsection{Protocol Flow}

Besides using a subset of the Salmon protocol to exchange messages,
Diaspora also uses WebFinger and hCard standards to discover identities
and fetch public profiles from remote servers. It is also possible to
use ActivityStreams and PubSubHubbub to provide public feeds.

The basic flow is to discover users, fetch their public information,
then create and send Salmon messages according to the desired
interaction. Diaspora pods will also send Salmon messages to subscribed
servers, notifying about publications and updates.

\subsection{Diaspora Current State}

Both the project and the protocol specification remain in active
development. The latest specification dates from July, 2016. There is
also a protocol implementation written as a Ruby library, used in the
main Diaspora project, that includes most of the message exchange
mechanism. That library can be reused by other projects that want to
implement federation with the Diaspora network.

The visibility gained by the Diaspora project has provided it with a
significant following, including adoption of its specifications by other
projects such as Friendica and Hubzilla\footnote{\url{http://friendi.ca/} and
\url{https://project.hubzilla.org}}, which support some level of
integration with Diaspora pods. That visibility was not enough, though,
to trigger a formal standardization process.
