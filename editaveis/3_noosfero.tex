\chapter{NOOSFERO}
\label{chapter:3}

O Noosfero é um \textit{software} livre para a construção de redes sociais e
colaborativas. Desenvolvido em Ruby on Rails e licenciado sob AGPL versão 3, o
projeto conta com desenvolvimento ativo.

Além dos mecanismos de interação social, o Noosfero também conta com um sistema de
gerenciamento de conteúdo, o que possibilita a criação de \textit{blogs} e o
compartilhamento de arquivos. A plataforma também pode ser estendida por
\textit{plugins} desenvolvidos pela comunidade, e conta com o conceito de ambientes,
que permitem a criação de diversas redes isoladas funcionando sobre uma mesma
instância da aplicação.

As informações e publicações de pessoas e organizações podem ser públicas ou
privadas. Já os relacionamentos entre estas entidades podem ser tanto simétricos
como assimétricos.

Enquanto um relacionamento simétrico depende da concordância de ambas as partes para
o compartilhamento das informações privadas (como por exemplo amizades ou
filiações), um relacionamento assimétrico depende apenas do interesse de uma das
entidades em acompanhar as informações públicas de algum perfil (como no caso da
funcionalidade de seguidores).


\subsection{Federação com outras redes sociais}
\label{subsec:federacao_externa}

A federação com redes não Noosfero deve usar a infraestrutura desenvolvida para a
integração entre redes Noosfero, principalmente os mecanismos de usuários externos.
Todas as atividades de implementação propostas neste trabalho foram desenvolvidas
com base no protocolo Diaspora, respeitando a conclusão alcançada pela comunidade.

A longo prazo, o ideal é que as mesmas funcionalidades implementadas na federação de
redes Noosfero sejam suportadas. É essencial que o Noosfero responda ao protocolo de
uma forma que permita a integração bidirecional, permitindo que outras redes
também sejam capazes de descobrir usuários e consumir publicações do Noosfero.

Ainda que o protocolo deva ser suportado com completude, neste primeiro momento é
interessante implementar apenas uma parte da especificação. O desenvolvimento de um
conjunto básico de funcionalidades, além de garantir um nível limitado de federação,
deve cobrir parte da reestruturação arquitetural necessária, ajudando na
identificação das modificações que devem ser introduzidas no Noosfero, o que vai ser
útil para discussões futuras.

As funcionalidades a seguir foram definidas para a primeira interação de
desenvolvimento definindo as contribuições deste trabalho.

\begin{enumerate}
  \item{Possibilitar um usuário possa acessar outras redes com as credenciais de sua
        rede de origem, sem a necessidade de um novo cadastro. Inicialmente, deve-se
        desenvolver um \textit{plugin} para autorização com OpenID, visto que é o
        único padrão implementado pelo Diaspora;}

  \item{Permitir que usuários de outras redes possam ser encontrados através da
        busca do Noosfero, o que é o primeiro passo para as inter-relações. A busca
        deve respeitar o padrão de descoberta do Diaspora, baseado no WebFinger;}

  \item{Implementar relações assimétricas entre os usuários do Noosfero e de redes
        que respondam ao protocolo Diaspora. O protocolo deve ser usado para que as
        duas redes estejam cientes da relação;}

  \item{Permitir que o Noosfero receba as publicações enviadas por redes que
        implementem o protocolo Diaspora.}
\end{enumerate}

\subsubsection{Implementação do Protocolo Diaspora}

O protocolo Diaspora está implementado no formato de uma \textit{gem}, que pode ser
facilmente incorporado como dependência em projetos desenvolvidos em linguagem Ruby,
como o Noosfero. A \textit{gem} é mantida pela mesma comunidade responsável pelo
projeto original, e sempre acompanha a última especificação adotada.

Segundo os mantenedores, o protocolo Diaspora ainda não está estável, e alterações
capazes de introduzir incompatibilidades podem ser introduzidas na \textit{gem}. A
implementação inicial no Noosfero deve ser executada sobre a última versão estável
disponível no momento de sua adição ao projeto.

O primeiro elemento necessário para a implementação da interoperabilidade é um
mecanismo de descoberta de informações entre servidores, no caso do Diaspora o
WebFinger. A implementação base já está disponível no Noosfero, sendo necessário
testar a integração com outro sistema que implemente o protocolo. Neste ponto a
especificação do Diaspora também exige que a resposta WebFinger inclua um
hCard\footnote{O hCard é um formato para a representação de informações de uma
entidade, como por exemplo uma pessoa. \url{http://microformats.org/wiki/hcard}},
que por sua vez contém informações pessoais de cada usuário, e também deve ser
implementado no Noosfero.

O segundo elemento é a comunicação entre os servidores, realizada através da troca
de mensagens contendo entidades, que representam as interações entre os usuários e
conteúdos. É importante que o Noosfero reconheça as entidades listadas a seguir.

\begin{itemize}
  \item{Perfil de usuário e atualizações}
  \item{Publicações e respostas}
  \item{Participação (inscrição em publicações)}
  \item{Contato entre usuários (seguir ou deixar de seguir)}
\end{itemize}

As mensagens são transferidas entre os servidores por meio do protocolo Salmon. A
\textit{gem diaspora\char`_federation} provê as funcionalidades de criptografia e
serialização necessárias para a comunicação sobre este padrão, e será adicionada
como dependência do Noosfero para auxiliar a implementação do protocolo Diaspora.

A implementação destes recursos também foi uma das contribuições deste trabalho, e
sua realização está descrita no Capítulo \ref{chapter:4}.
