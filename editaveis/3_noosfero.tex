\chapter{NOOSFERO}
\label{chapter:3}

O Noosfero é um \textit{software} livre para a construção de redes sociais e
colaborativas. Desenvolvido em Ruby on Rails e licenciado sob AGPL versão 3, o
projeto conta com desenvolvimento ativo.

Além dos mecanismos de interação social, o Noosfero também conta com um sistema de
gerenciamento de conteúdo, o que possibilita a criação de \textit{blogs} e o
compartilhamento de arquivos. A plataforma também pode ser estendida por
\textit{plugins} desenvolvidos pela comunidade, e conta com o conceito de ambientes,
que permitem a criação de diversas redes isoladas funcionando sobre uma mesma
instância da aplicação.

As informações e publicações de pessoas e organizações podem ser públicas ou
privadas. Já os relacionamentos entre estas entidades podem ser tanto simétricos
como assimétricos.

Enquanto um relacionamento simétrico depende da concordância de ambas as partes para
o compartilhamento das informações privadas (como por exemplo amizades ou
filiações), um relacionamento assimétrico depende apenas do interesse de uma das
entidades em acompanhar as informações públicas de algum perfil (como no caso da
funcionalidade de seguidores).



\section{SUPORTE À FEDERAÇÃO}

Já existe uma iniciativa de federação no Noosfero em desenvolvimento por parte da
comunidade \footnote{A discussão a respeito da federação no Noosfero está registrada
no endereço a seguir
\url{https://softwarepublico.gov.br/gitlab/noosferogov/noosfero/wikis/federacao}.},
tendo o autor deste trabalho colaborado desde então. O objetivo é possibilitar a
integração tanto com outras instâncias do Noosfero como com outras redes sociais, o
que exige a adoção de especificações que tenham o mínimo de aderência na comunidade.
A utilização de padrões difundidos amplia as possibilidades de integração dentre
outras redes sociais federadas.

Antes da execução deste trabalho, os protocolos Diaspora e OStatus já haviam sido
escolhidos como referência para a implementação da federação no Noosfero, resultado
de uma observação das discussões e execução de projetos como o Hubzilla, Friendica e
o próprio Diaspora.

As primeiras contribuições com a federação no Noosfero tiveram início antes deste
trabalho, e estão descritas na Subseção \ref{subsec:federacao_noosfero}. Os
resultados relativos à integração com outras redes, apresentados na Subseção
\ref{subsec:federacao_externa}, são produtos deste projeto.


\subsection{Federação entre redes Noosfero}
\label{subsec:federacao_noosfero}

As atividades de implementação de federação já desenvolvidas podem ser separadas em
quatro fases, que cumpriram objetivos distintos de integração, que cobrem desde as
funcionalidades até a reestruturação da arquitetura da aplicação.

Foi definido que um usuário de uma rede Noosfero pode acessar qualquer outra
instância com as credenciais de sua rede de origem. Um usuário federado deve ser
capaz de visualizar conteúdos públicos, comentar publicações, seguir usuários, e
deixar mensagens em murais. As notificações destas interações devem ser entregues
tanto aos usuários na rede local, quanto ao usuário na rede de origem.

O protocolo construído entre redes Noosfero é baseado nas especificações do
WebFinger e OAuth para a descoberta de identidade e autorização de perfis,
respectivamente. Em relação à comunicação entre as redes, o protocolo Diaspora foi
definido como referência.

\subsubsection{Fase 1: preparação}

Até a versão 1.5 do Noosfero, todos os relacionamentos entre as entidades da rede
eram baseados no conceito de relacionamento simétrico. No entanto, as demais redes
federadas, e a maioria dos padrões mais implementados, trabalham apenas com o
conceito de relacionamentos assimétricos, o que incentivou o desenvolvimento da
funcionalidade de seguidores no Noosfero.

Na fase de preparação foram introduzidos os relacionamentos assimétricos através
desta funcionalidade. Os seguidores são notificados a respeito de atividades
públicas de perfis seguidos. No Noosfero, cada perfil pode permitir ou não que
usuários o sigam. Usuários por sua vez organizam os perfis seguidos em círculos,
categorizando suas relações.

\subsubsection{Fase 2: intercomunicações}

Durante a fase de intercomunicações foi construída a infraestrutura básica para a
integração entre redes Noosfero. Ambientes e usuários externos foram introduzidos à
arquitetura do Noosfero, que passa a suportar ações de usuários que não possuem
perfis locais.

O conceito de usuário externo introduzido nesta fase é importante para toda a
implementação da federação do Noosfero. Um usuário local do Noosfero é definido por
basicamente dois objetos de negócio --- um usuário, que armazena as credenciais de
acesso, e um perfil, que armazena suas demais informações na aplicação.

Um usuário externo não possui credenciais de acesso na instância visitada, apenas um
objeto que representa o seu perfil externo, e que do ponto de vista da implementação
deve ser capaz de reproduzir o comportamento de um perfil comum. A implementação
alcançada faz uso de métodos \textit{stub} e relações polimórficas para a reprodução
do comportamento.

Nesta fase também foi implementada a especificação do WebFinger, que já está sendo
utilizada para a descoberta de usuários na autenticação entre redes Noosfero.

Inicialmente, apenas as redes listadas no diretório central do Noosfero
\footnote{\url{directory.noosfero.org}} podem ser habilitadas no painel de
administração da federação. A descentralização desta lista ou a automatização do
processo de descoberta não fizeram parte do planejamento inicial.

\subsubsection{Fase 3: integração externa}

A fase de integração externa teve como objetivo aproveitar a infraestrutura de
usuários externos para autenticar usuários de outros serviços sem a necessidade de
perfis locais. Com isto, usuários de sistemas que suportem OAuth podem acessar o
Noosfero, consumir conteúdo, e executar um conjunto limitado de ações.

Durante esta etapa, o \textit{plugin} que torna o Noosfero em um cliente OAuth foi
evoluído para permitir que usuários possam tanto criar um perfil local a partir das
informações da rede de origem, como também apenas acessar o Noosfero com um perfil
temporário. Por enquanto, os únicos fornecedores OAuth suportados são o Google,
Facebook, Twitter, GitHub, e o próprio Noosfero. No entanto, novos fornecedores
podem ser facilmente adicionados.

\subsubsection{Fase 4: inter-relações}

A última fase de desenvolvimento da federação de redes Noosfero foi implementar o
relacionamento entre usuários de instâncias diferentes. De modo geral, esta fase
consistiu em permitir que usuários externos sejam capazes de seguir perfis, comentar
conteúdos, e publicar em murais de outros usuários.

De forma a permitir relações entre usuários federados foi necessário refatorar a
funcionalidade de seguidores, adicionando o suporte a perfis externos. Neste ponto,
usuários federados podem tanto seguir usuários locais, quanto serem seguidos por
eles.

Essa fase também envolve a implementação da infraestrutura de troca de mensagens,
que seria utilizada nas notificações e interações entre os usuários. Até a
execução deste trabalho esse mecanismo não foi completamente definido.


\subsection{Federação com outras redes sociais}
\label{subsec:federacao_externa}

A federação com redes não Noosfero deve usar a infraestrutura desenvolvida para a
integração entre redes Noosfero, principalmente os mecanismos de usuários externos.
Todas as atividades de implementação propostas neste trabalho foram desenvolvidas
com base no protocolo Diaspora, respeitando a conclusão alcançada pela comunidade.

A longo prazo, o ideal é que as mesmas funcionalidades implementadas na federação de
redes Noosfero sejam suportadas. É essencial que o Noosfero responda ao protocolo de
uma forma que permita a integração bidirecional, permitindo que outras redes
também sejam capazes de descobrir usuários e consumir publicações do Noosfero.

Ainda que o protocolo deva ser suportado com completude, neste primeiro momento é
interessante implementar apenas uma parte da especificação. O desenvolvimento de um
conjunto básico de funcionalidades, além de garantir um nível limitado de federação,
deve cobrir parte da reestruturação arquitetural necessária, ajudando na
identificação das modificações que devem ser introduzidas no Noosfero, o que vai ser
útil para discussões futuras.

As funcionalidades a seguir foram definidas para a primeira interação de
desenvolvimento definindo as contribuições deste trabalho.

\begin{enumerate}
  \item{Possibilitar um usuário possa acessar outras redes com as credenciais de sua
        rede de origem, sem a necessidade de um novo cadastro. Inicialmente, deve-se
        desenvolver um \textit{plugin} para autorização com OpenID, visto que é o
        único padrão implementado pelo Diaspora;}

  \item{Permitir que usuários de outras redes possam ser encontrados através da
        busca do Noosfero, o que é o primeiro passo para as inter-relações. A busca
        deve respeitar o padrão de descoberta do Diaspora, baseado no WebFinger;}

  \item{Implementar relações assimétricas entre os usuários do Noosfero e de redes
        que respondam ao protocolo Diaspora. O protocolo deve ser usado para que as
        duas redes estejam cientes da relação;}

  \item{Permitir que o Noosfero receba as publicações enviadas por redes que
        implementem o protocolo Diaspora.}
\end{enumerate}

\subsubsection{Implementação do Protocolo Diaspora}

O protocolo Diaspora está implementado no formato de uma \textit{gem}, que pode ser
facilmente incorporado como dependência em projetos desenvolvidos em linguagem Ruby,
como o Noosfero. A \textit{gem} é mantida pela mesma comunidade responsável pelo
projeto original, e sempre acompanha a última especificação adotada.

Segundo os mantenedores, o protocolo Diaspora ainda não está estável, e alterações
capazes de introduzir incompatibilidades podem ser introduzidas na \textit{gem}. A
implementação inicial no Noosfero deve ser executada sobre a última versão estável
disponível no momento de sua adição ao projeto.

O primeiro elemento necessário para a implementação da interoperabilidade é um
mecanismo de descoberta de informações entre servidores, no caso do Diaspora o
WebFinger. A implementação base já está disponível no Noosfero, sendo necessário
testar a integração com outro sistema que implemente o protocolo. Neste ponto a
especificação do Diaspora também exige que a resposta WebFinger inclua um
hCard\footnote{O hCard é um formato para a representação de informações de uma
entidade, como por exemplo uma pessoa. \url{http://microformats.org/wiki/hcard}},
que por sua vez contém informações pessoais de cada usuário, e também deve ser
implementado no Noosfero.

O segundo elemento é a comunicação entre os servidores, realizada através da troca
de mensagens contendo entidades, que representam as interações entre os usuários e
conteúdos. É importante que o Noosfero reconheça as entidades listadas a seguir.

\begin{itemize}
  \item{Perfil de usuário e atualizações}
  \item{Publicações e respostas}
  \item{Participação (inscrição em publicações)}
  \item{Contato entre usuários (seguir ou deixar de seguir)}
\end{itemize}

As mensagens são transferidas entre os servidores por meio do protocolo Salmon. A
\textit{gem diaspora\char`_federation} provê as funcionalidades de criptografia e
serialização necessárias para a comunicação sobre este padrão, e será adicionada
como dependência do Noosfero para auxiliar a implementação do protocolo Diaspora.

A implementação destes recursos também foi uma das contribuições deste trabalho, e
sua realização está descrita no Capítulo \ref{chapter:4}.
