\chapter{CONSIDERAÇÕES PRELIMINARES}
\label{chapter:5}

Através da implementação descrita no Capítulo \ref{chapter:4} é possível obter um
conjunto de funções que demonstra as capacidades de federação através do protocolo
Diaspora. Todas as funções listadas abaixo foram resultado da contribuição realizada
durante a execução deste trabalho.

\begin{itemize}
  \item{Conhecendo o identificador do usuário e o endereço de sua rede de origem, é
        possível encontrá-lo a partir da busca de pessoas;}
  \item{A partir da rede local, é possível seguir o perfil de um usuário em sua rede
        de origem, expressando a intenção de receber suas publicações;}
  \item{As publicações de usuários remotos podem ser visualizadas em seu perfil
        local do Noosfero assim que seu servidor de origem expeça as notificações.}
\end{itemize}

Essas funcionalidades tornam o Noosfero federado com qualquer aplicação que
implemente o protocolo Diaspora. Mais precisamente, por enquanto o Noosfero só
depende de uma aplicação que respeite a especificação no suporte à descoberta de
usuários, recebimento de contatos, e envio notificações de publicação.

Algumas destas funções também dependem do suporte no Noosfero aos padrões WebFinger
e hCard, o que já possibilita que usuários do Noosfero possam ser descobertos por
qualquer outra rede que implemente o mesmo protocolo, inclusive outras instâncias do
Noosfero. Isso permite que a implementação através do Diaspora substitua a federação
entre redes Noosfero implementada anteriormente através da API, o que não seria
custoso, já que o código ainda não foi integrado na \textit{branch} primária.

Para implementar este subconjunto do protocolo Diaspora, foi necessário introduzir
algumas modificações que afetaram a arquitetura do Noosfero. A lista abaixo descreve
as mudanças que exerceram maior impacto sobre o código.

\begin{itemize}
  \item{A busca passou a retornar perfis de usuários externos;}
  \item{Usuários remotos possuem uma página de perfil que pode ser visualizada na
        instância local, e devem ser capazes de publicar \textit{scraps};}
  \item{Todas as classes de domínio envolvidas na federação devem possuir um
        identificador universal (GUID). Por enquanto, o novo campo foi adicionado
        nas classes que representam usuários remotos e \textit{scraps};}
  \item{Cada um dos usuários do Noosfero possui um par de chaves RSA próprio para o
        envio de mensagens privadas através do Salmon;}
  \item{Usuários remotos podem ser seguidos e adicionados em círculos por usuários
        locais;\footnote{Essa alteração em específico já havia sido desenvolvida
        como parte da federação entre redes Noosfero. No entanto, até este ponto a
        contribuição ainda está sob revisão, e por esse motivo foi integrado
        manualmente à \textit{branch} em que a federação com o Diaspora foi
        desenvolvida.}}
\end{itemize}

Tendo sumarizado as contribuições deste trabalho, é possível propor um cronograma de
referência para a continuidade do projeto. A Tabela \ref{tab:cronograma} exibe um
cronograma mensal contendo os marcos para a finalização da federação com o Diaspora.
Também propõe uma última atividade para a integração entre duas instâncias do
Noosfero --- o Rede Comunidade UnB e o Stoa Social da USP, oferecendo um caso de
federação entre dois sistemas em produção.

\begin{table}[h]
  \begin{center}
    \begin{tabular}{| l | p{12cm} |}
    \hline
    \textbf{Mês} & \textbf{Atividades} \\
    \hline
    Janeiro & implementar retrações das entidades já implementadas, permitindo que
    o Noosfero remova publicações apagadas na rede original. Também é necessário
    tratar atualizações de perfis. \\
    \hline
    Fevereiro & Implementar o suporte às entidades de Contato no Noosfero, fazendo
    com que seus usuários também possam ser seguidos a partir de outras redes. Nesse
    ponto, é necessário estabelecer o mecanismo que envia novas publicações a todos
    os inscritos. \\
    \hline
    Março & Federar conteúdos privados, como publicações para círculos específicos,
    ou mensagens diretas. \\
    \hline
    Abril e Maio & Analisar a integração do código do protocolo Diaspora na
    \textit{branch} de federação do \textit{core} do Noosfero, considerando a
    substituição da implementação atual. \\
    \hline
    Junho & Atualizar o Noosfero do Comunidade UnB e do Stoa para uma versão
    federada, e realizar a integração entre as plataformas. \\
    \hline
    \end{tabular}
  \end{center}
  \caption{Cronograma de referência para a continuidade do projeto}
  \label{tab:cronograma}
\end{table}
