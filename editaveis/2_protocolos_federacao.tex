\chapter{PLATAFORMAS E PROTOCOLOS DE FEDERAÇÃO}

Uma das motivações de sistemas federados trata da descentralização de informações, ao
mesmo tempo em que são oferecidos mecanismos necessários para a integração de
diversas fontes de conteúdo. Em consequência disto, a federação de sistemas provoca
impacto sobre a liberdade no controle de informações e privacidade dos usuários.

Enquanto estes argumentos não fazem sentido em uma rede social popular, em que os
serviços sejam controlados por uma única entidade, são significativos para redes que
podem ser mantidas por entidades independentes, o que engloba aplicações de
\textit{software} livre. Considerando que qualquer rede social livre pode contar com
diversos números de instâncias mantidas por entidades distintas, os benefícios da
federação são de grande interesse da comunidade.

Apesar do interesse em adicionar suporte à federação em sistemas abertos, não existe
um padrão reconhecido pela comunidade de desenvolvimento [[[referencia?]]]... Como
consequência, as aplicações atualmente federadas abordam os desafios intermediários
do fluxo de integração através de tecnologias já difundidas, como \textit{feeds},
\textit{APIs}, e mecanismos de troca de mensagens.

A ausência de uma especificação leva os desenvolvedores a adotarem estratégias
diferentes na implementação da federação, ainda que utilizando os mesmos mecanismos.
Por outro lado, algumas implementações adquirem certa popularidade, tornando-se
especificações de referência para outros sistemas, e garantindo um certo nível de
interoperabilidade.

O OStatus é um exemplo de especificação que, apesar de ter surgido como referência
para o sistema que foi a se tornar o GNU Social, ganhou popularidade e também foi
implementado por alguns outras aplicações, como o Friendica. Outro exemplo é o
protocolo Diaspora, inicialmente implementado por uma rede social de mesmo nome, mas
que acabou ganhando apelo assim que o sistema se tornou um exemplo popular de rede
completamente federada, passando a ser implementado por outros sistemas como o
Hubzilla.

Ao discutir federação no contexto de sistemas livres é interessante analisar as
especificações populares e cada um dos mecanismos utilizados nas implementações de
referência. Compreendendo os protocolos existentes é possível indicar o estado das
iniciativas de sistemas federados, o que leva ao estado e às necessidades de
interoperabilidade.

\section{OStatus}

O OStatus é uma especificação baseada em um conjunto de protocolos construída com a
intenção de oferecer uma estrutura completa para a padronização e interação entre
redes sociais distintas. Foi inicialmente proposto por Evan Prodromou e implementado
no StatusNet, que posteriormente deu origem ao projeto Gnu social.

Apesar de Atom e RSS constituírem padrões corriqueiros para a implementação de feeds
na internet, tratam-se de especificações que não são capazes de permitir interações
em tempo real. A ideia do OStatus é combinar \textit{feeds} Atom com uma série de
mecanismos capazes de permitir identidade, confiabilidade na comunicação entre os
servidores.

O projeto conta com uma das maiores iniciativas de padronização de protocolos de
federação, tornando-se alvo de um grupo de trabalho da W3C desde 2012. Apesar de não
ter apresentado maiores avanços nos últimos anos, ainda há interesse da comunidade em
mantê-lo e utilizá-lo. 

\subsection{PubSubHubbub}

De forma a possibilitar as interações em tempo real previstas pelo  OStatus, é
necessário utilizar algum tipo de mecanismo que, em combinação com o Atom, possa
entregar notificações 

O PubHubSub é um sistema de publicações de publicação e assinatura distribuído.
Possibilita que servidores se inscrevam em \textit{feeds}, e sejam notificados assim
que alguma alteração aconteça. A ideia é que uma série de serviços se inscrevem em
diretórios centrais (ou \textit{hubs}), expressando interesse em receber notificações
de atualização. Serviços inscritos devem identificar o tópico desejado através de
URLs, e oferecer um servidor disponível pela internet para que a notificação possa
ser realizada. Entidades interessadas em publicar algum tipo de conteúdo incluem uma
referência ao \textit{hub}. É de responsabilidade dos produtores de conteúdo
notificar o \textit{hub} em novas publicações, que por sua vez são responsáveis por
divulgar para cada um dos indivíduos inscritos.

Sem entrar em maiores detalhes a respeito da especificação, a comunicação dos
\textit{hubs} com as entidades que publicam e consomem os conteúdos acontecem sobre
HTTP, e identificadas por endereços URL. As notificações são baseadas na execução de
trechos arbitrários de código ativados a partir de \textit{endpoints} URL, seguindo a
ideia de \textit{webhooks}.

\subsection{WebFinger}

% referenciar a RFC 7033

O WebFinger é um protocolo de descoberta de identidade que soluciona o problema do
compartilhamento das informações de usuários entre servidores remotos. A proposta é
que a partir de um atributo identificador de usuário e do endereço do seu servidor de
origem, seja possível garantir a existência do usuário e obter suas informações
públicas.

A especificação do WebFinger propõe que todos os recursos possam ser identificados
por uma URL, e que todas as solicitações e respostas sejam realizadas através de
requisições HTTP. Servidores que forneçam informações através do WebFinger devem
responder aos \textit{endpoints} definidos no protocolo com objetos JSON. Servidores
que pretendam consumir tais informações precisam conhecer apenas a URL do recurso de
interesse, o que pode ser encontrado a partir do identificador do usuário e do
domínio do servidor de origem (em um processo que pode ser classificado como LRDD). 

% referenciar LRDD: https://tools.ietf.org/html/draft-hammer-discovery-06 

No caso específico do OStatus, o WebFinger é utilizado no início de cada interação
entre usuários. Por exemplo, considerando um cenário em que um usuário pretende se
inscrever no \textit{feed} de um usuário remoto, conhecendo seu identificador e o
servidor de origem. Antes de solicitar a inscrição através de um servidor PubHubSub e
notificar o usuário remoto, o usuário local deve primeiro obter as informações
necessárias (como por exemplo a URL do \textit{feed} público) através de uma
requisição  WebFinger.

\subsection{Activity Streams}

O Activity Streams é uma especificação que propõe a padronização da representação de
atividades em redes sociais que tenham interesse em oferecer mecanismos de
integração. O emprego deste protocolo é justificado pela necessidade em registrar as
atividades do usuário no interior da rede, bem como pela representação através de um
formato que possa ser consumido por qualquer serviço interessado e autorizado.

A especificação mais atual do Activity Streams exige uma mensagem em formato JSON,
incluindo uma série de propriedades que identificam a ação, o usuário responsável, e
a entidade alvo. Considera-se ainda o envio de atividades complexas que contenham
referências a outras atividades ou entidades do sistema. 

Além da especificação que adota o padrão JSON, outras especificações utilizam o
padrão Atom, com restrições semelhantes em relação ao conteúdo e formato das
mensagens. O OStatus adota o protocolo especificado em Atom para o envio de entradas
de Activity Streams.

\subsection{Salmon}

Combinar soluções como Atom e PubHubSub permitem que conteúdos sejam publicados e
atualizados em tempo real. No entanto, a partir do ponto em que um conteúdo pode ser
consumido a partir de um número arbitrário de serviços, deve-se investir esforço para
garantir que o estado da discussão (como comentários e avaliações relacionados a uma
publicação) seja o mesmo em toda a rede. Mais especificamente, qualquer entidade
consumidora que atualize o estado da discussão deve notificar o servidor fonte, que
de alguma forma deve garantir um estado uniforme para o restante da rede.

O Salmon é capaz de garantir a unificação da discussão entre a rede em que um
conteúdo foi publicado, e todos os serviços consumidores. Trata-se de uma
especificação de troca de mensagens que define como as entidades consumidoras devem
notificar os servidores fonte a respeito da atualização do estado de discussões, e
como deve estes devem reagir a tais eventos.

Servidores interessados em implementar o protocolo devem incluir um \textit{endpoint}
Salmon em seus \textit{feeds}, que será lido pelas entidades consumidoras, e para o
qual as mensagens devem ser enviadas sobre HTTP. Uma mensagem Salmon se trata de uma
entrada adicional no \textit{feed}, codificada em \textit{base64}, envolta por outra
estrutura XML assinada digitalmente. Ao Receber a mensagem, o servidor possui a
liberdade de proceder conforme suas próprias políticas, desde que retorne uma
resposta HTTP válida.

\subsection{Fluxo do OStatus}

Cada protocolo utilizado na especificação do OStatus possui uma responsabilidade
específica no fluxo de comunicação. A [[[referenciar o diagrama]]] apresenta um
cenário em que dois usuários de servidores distintos interagem através deste fluxo,
apontando o papel de cada um dos mecanismos intermediários.

% diagrama do workflow do OStatus

\begin{enumerate} \item{O usuário A descobre o endereço do servidor de origem do
usuário B através do WebFinger} \item{O usuário A se inscreve no \textit{feed}
público através do PubHubSub através do endereço obtido} \item{O usuário A envia uma
nova entrada no formato Activity Streams através de uma mensagem Salmon para o
usuário B, notificando-o da ação} \item{O servidor do usuário B atualiza sua lista de
seguidores. Eventualmente, o usuário B publica um novo conteúdo, notificando os
inscritos através do PubHubSub} \item{O usuário A recebe o \textit{feed} atualizado
através do PubHubSub, e decide publicar um novo comentário. Uma nova entrada no
\textit{feed} é criada em formato Activity Streams, e novamente enviada ao servidor
original através de uma mensagem Salmon} \item{O servidor do usuário B recebe a
mensagem, e registra um novo comentário, atualizando seu \textit{feed} e notificando
novamente os seus inscritos} \end{enumerate}


\section{Diaspora}

O projeto Diaspora surgiu com a intenção de implementar o conceito de redes sociais
descentralizadas em resposta aos problemas de liberdade e privacidade encontrados em
plataformas sociais privadas. Sua primeira versão foi lançada em Setembro de 2010
como fruto de uma campanha de financiamento coletivo, passando a ser completamente
governado pela comunidade a partir de Agosto de 2012.

% Descrever melhor o que é o Diaspora

A proposta dos desenvolvedores é evitar a centralização de conteúdo, portanto a falta
de controle sobre as informações, construindo uma rede de instâncias pessoais da
plataforma, ou \textit{pods}. Cada servidor Diaspora reúne as informações dos seus
usuários, mas em cooperação com outros servidores, possibilita a interação e
compartilhamento de informações entre usuários de diferentes \textit{pods}.

Além da própria plataforma, desenvolvida em Ruby on Rails, uma biblioteca que
implementa o protocolo de federação do Diaspora também está disponível no formato de
uma \textit{gem} Ruby. Todas as soluções são distribuídas sob a licença AGPL versão
3. 

O protocolo de federação do Diaspora foi definido para convergir com o OStatus assim
que o último passe a suportar o conceito de privacidade limitada. A implementação
leva em consideração a troca de mensagens em formato XML, respeitando alguns
conceitos básicos como a existência de usuários remotos e atualização remota e
retransmissão.

\subsection{Usuários Remotos}

Um conceito fundamental para a implementação de redes federadas é considerar a
existência de usuários remotos. A maioria das aplicações só possui o conceito de
usuários locais, que estão diretamente autenticados no serviço e possuem todas as
informações na base de dados local. No entanto, ao possibilitar a interação com
usuários de outras redes, usuários externos ao sistema precisam ser explicitamente
considerados na implementação das funcionalidades.

O Diaspora conceitualiza seus usuários em locais e externos. Enquanto usuários locais
respeitam a definição tradicional, os usuários externos interagem com a aplicação
através dos mecanismos de federação. É importante prever a existência de usuários
externos na modelagem de sistemas. Por este motivo, implementar federação em
aplicações já consolidadas pode exigir um certo esforço de refatoração. % ref?

\subsection{Capacidade de Retransmissão}

A retransmissão é essencial em sistemas federados, visto que interações em uma rede
eventualmente devem afetar \textit{pods} relacionados. A restrição implementada pelo
Diaspora indica que todas as notificações neste contexto sejam entregues tanto aos
usuários locais quanto aos usuários remotos. Adicionalmente, a notificação de
usuários locais não deve depender da resposta dos demais \textit{pods}.

Em configurações de integração mais complexas, a capacidade de retransmissão passa a
ser um requisito essencial para a troca de mensagens. Considere uma situação
hipotética em que \textit{pods} \textbf{A} e \textbf{B} são federados com o
\textit{pod} \textbf{C}, mas não entre si. Qualquer modificação em um conteúdo de
\textbf{C} compartilhado com \textbf{A} e \textbf{B} deve afetar os três
\textit{pods}. No entanto, se a modificação partir de \textbf{A}, há uma dificuldade
em notificar \textbf{B}, visto que o \textit{pod} em questão só reconhece a
existência de \textbf{C}. A solução defendida pela implementação do Diaspora é que
\textbf{C} retransmita a notificação para todos os \textit{pods} com os quais o
conteúdo seja compartilhado. Isso garante que todos os sistemas federados envolvidos
em uma interação sejam notificados, contribuindo com consistência das informações.

\subsection{Troca de Mensagens}

O Diaspora define um conjunto de mensagens que delimitam as possíveis interações
entre \textit{pods}.

\begin{itemize} \item{Compartilhamento de informações} \item{Publicações de conteúdo}
\item{Comentários e reações a publicações} \item{Mensagens privadas} \end{itemize}

A troca de mensagens segue a definição do protocolo Diaspora, que utiliza um
subconjunto do protocolo Salmon. De modo geral, restringe como a mensagem deve ser
construída e enviada para o \textit{endpoint} Salmon do \textit{pod} de destino. 

\subsection{Fluxo do Protocolo Diaspora}

O protocolo Diaspora cobre tanto a descoberta de identidades entre os servidores,
quanto o envio e recebimento de informações. De forma análoga ao OStatus, parte da
implementação também utiliza outros protocolos, como o WebFinger, o Activity Streams
e o Salmon. Uma outra especificação implementada é o hCard, que trata da
representação das informações de um usuário em formato HTML ou XML. % ref hCard

O diagrama da [[[figura]]] representa o fluxo de uma interação entre dois usuários de
\textit{hubs} distintos.

% diagrama do workflow do Diaspora

% revisar o fluxo do protocolo \begin{enumerate} \item{O usuário A deseja seguir o
\textit{feed} do usuário B localizado em outro \textit{hub} do Diaspora. O primeiro
passo é descobrir o WebFinger de B através das informações de seu \textit{hub}, e
solicitar o seu perfil} \item{O \textit{hub} do usuário B responde com o perfil
WebFinger de B, que além de suas informações inclui uma URL do seu \textit{feed} em
Activity Streams, e uma chave pública RSA para a troca de mensagens} \item{O usuário
A cria uma nova mensagem solicitando o compartilhamento e a envia por meio de Salmon
para o usuário B} \item{O usuário B começa a compartilhar suas publicações com A,
enviando uma nova mensagem Salmon a cada nova publicação} \item{O usuário A recebe a
nova publicação, e decide enviar um novo comentário.  uma nova mensagem Activity
Streams é construída e enviada por meio de Salmon para o \textit{hub} original}
\item{O usuário B recebe o comentário de A, e notifica todos os outros \textit{hubs}
inscritos na mesma publicação} \end{enumerate}


\section{Demais Iniciativas}
