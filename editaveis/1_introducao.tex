\chapter{INTRODUÇÃO}

% introduzir mídias sociais

% definir privacidade
% propriedade das informações

% apresentar as críticas às redes sociais "grandes"



\section{REDES DE COMUNICAÇÃO DESCENTRALIZADAS}

% Tipos de redes de comunicação...
% definir redes descentralizadas, distribuídas, federadas...

% Redes descentralizadas e a internet
% A "centralização" da internet com os grandes provedores
% por que descentralizar novamente? (fazer um paralelo com serviços)



\section{REDES SOCIAIS FEDERADAS}

% {aproveitar o gancho do capítulo passado: busca por descentralizar)
% Definição de federação de sistemas , contextualização do problema a que se propõe

% exemplificar sistemas federados com o e-mail

% importância de padrões
% interoperabilidade

% (portais de quarta geração...)...
% the five touchpoints: Gen-4 Portal Functionality: From Unification to Federation

% redes sociais federadas: como resolvem os problemas apresentados antes?

% - compartilhamento de recursos (consistência)
% - gerenciamento de identidades



\section{OBJETIVOS}
