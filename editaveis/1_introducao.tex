\chapter{INTRODUÇÃO}

Pode-se definir mídias sociais como aplicações da Internet que permitem que
indivíduos mantenham um conjunto de conexões com outros usuários, e compartilhem
informações pessoais e conteúdos gerados através de redes digitais \cite{boyd2007}.
Tais redes sociais são baseadas no fluxo de informações que nem sempre são públicas,
e geralmente acontece através de infraestruturas privadas de entidades interessadas
em prover o serviço, como por exemplo o Facebook e o Google. 

Em segurança computacional, pode-se definir privacidade como a capacidade de um
indivíduo controlar quais e como as informações relacionadas a ele podem ser
consumidas ou armazenadas, e para quais indivíduos estes dados podem ser divulgados.
\cite{stallings2010}. Além de garantir o direito à privacidade de seus usuários,
as redes sociais devem garantir a confidencialidade das informações que hospeda,
assegurando que dados privados são sejam expostos a indivíduos não autorizados.

Argumenta-se que o poder de um único grande provedor sobre um fluxo de dados
privados coloca em risco a privacidade de seus usuários, já que não garante a
confidencialidade das informações privadas. As críticas a estes provedores
geralmente é acentuada pela falta de transparência da utilização dessas informações,
que geralmente assume um papel comercial, como na identificação de padrões de
comportamento em suporte à publicidade. 

Uma alternativa à centralização do fluxo de dados privados é baseada no conceito
de redes descentralizadas, um dos padrões de organização de redes de comunicação.
Indica-se que em relação à organização de seus nós, redes de comunicação podem
assumir uma organização centralizada ou distribuída \cite{baran1964}. Como pode ser
visto na Figura \ref{fig:org_redes}, em contraste a redes centralizadas, uma rede
distribuída não depende de um único servidor central.

\begin{figure}[h]
	\centering
		\includegraphics[keepaspectratio=true,scale=0.6]{figuras/org_redes.eps}
	\caption{Tipos de redes de comunicação \cite{baran1964}}
	\label{fig:org_redes}
\end{figure}

Redes sociais privadas se assemelham à configuração de redes centralizadas, onde o
nó central representa o servidor, e cada um dos conectados um dos milhões de
clientes. Uma alternativa comum a esta configuração é a organização descentralizada,
que substitui um único nó central com servidores intermediários, conectados entre si
para o estabelecimento intercomunicação.

Alguns autores também definem esta organização como uma rede federada, que é similar
à definição de redes descentralizadas de \cite{baran1964}, mas também é definida na
literatura como um conjunto de implementações interoperáveis que respeitam o modelo
cliente-servidor \cite{barocas2012}. Esta definição é importante por que generaliza
o conceito de federação para outros tipos de sistemas de comunicação que não redes
literais de computadores, e indica a propriedade de extensibilidade --- qualquer
entidade capaz que garanta a interoperabilidade pode ser parte da federação.

A distribuição dos serviços contrapõe a existência de um único provedor, comum em
redes sociais fechadas como o Facebook e Twitter. A descentralização garante que o
armazenamento das informações não está restrito a apenas um proprietário. Mais
importante, a independência de extensão permite que qualquer indivíduo inclua um
novo servidor na federação, desde que respeite os critérios de interoperabilidade,
hospedando seus próprios dados. A descentralização do fluxo de dados privados
distribui a responsabilidade pela manutenção de confidencialidade, o que passa a não
depender de uma única entidade com intenções inexplícitas.

Por outro lado, a federação é fundada na capacidade de interoperabilidade entre
sistemas. Visto que o cenário provável envolve a comunicação entre sistemas
distintos em vários aspectos, é essencial considerar protocolos e padrões.

O objetivo deste trabalho é estudar a federação no contexto de redes sociais,
investigando quais os aspectos envolvidos na interoperabilidade destas mídias, e o
estado do esforço de padronização da interoperabilidade. Um segundo objetivo é
aprofundar esta análise no caso específico do Noosfero, uma rede social livre,
projetando e implementando uma prova de conceito de federação.

O Capítulo \ref{chapter:2} introduz a utilização de padronização de protocolos
de comunicação na interoperabilidade de sistemas, aborda a utilização de
especificações na federação de sistemas, e apresenta as iniciativas de
interoperabilidade e padronização neste contexto.

O Capítulo \ref{chapter:3} apresenta o Noosfero e descreve o estado da federação
até a publicação deste trabalho. Neste ponto, propõe-se a evolução da federação com
outras redes, apresentando uma proposta de implementação com base no protocolo
Diaspora.

Por fim, o Capítulo \ref{chapter:4} apresenta os resultados da implementação do
protocolo Diaspora no Noosfero, descrevendo as atividades realizadas para adicionar
o suporte à especificação, e as considerações que devem ser levadas em consideração
na continuidade deste trabalho.
