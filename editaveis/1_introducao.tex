\chapter{INTRODUÇÃO}


\section{ARQUITETURAS DE REDES DE COMUNICAÇÃO}

% Tipos de redes de comunicação...

Redes de comunicação podem assumir diferentes arquiteturas, de acordo com a 
estruturação de seus nós. De acordo com (ref to Baran) redes podem assumir
configurações centralizadas, distribuídas, e descentralizadas. 

[--- imagem das três categorias de rede segundo Baran ---]

Outros autores propõem definições diferentes para redes distribuídas e
descentralizadas. usa o termo descentralização para compor os
conceitos de nós [--- descrever os conceitos aqui ---]. 

% Apresentar mais conceitos arquiteturais de redes.

\subsection{Redes de comunicação descentralizadas}

Por mais que existam diferentes definições para a arquitetura descentralizada de
redes de comunicação, o conceito converge para uma organização de nós que, apesar
de respeitar a noção de cliente-servidor, também é formada por um conjunto de
servidores que se comunicam de maneira distribuída. A Internet, por exemplo, surgiu
como um rede descentralizada onde um conjunto de clientes acessa, através de um
servidor local, uma rede maior de servidores distribuídos.

Apesar da Internet ser, por natureza, uma rede descentralizada, a popularização de 
grandes serviços e plataformas integradoras tende à adoção de uma estrutura  cada
vez mais centralizada. % Encontrar números que comprovam essa afirmação 

% benefícios de redes centralizadas
% centralização e descentralização da internet
% tendência para a centralização, apresentar os problemas de privacidade
% por que descentralizar novamente? (fazer um paralelo com serviços)

% aplicações distribuídas (kurose página 32)


\section{SISTEMAS FEDERADOS}

% {aproveitar o gancho do capítulo passado: busca por descentralizar)
% Definição de federação, contextualização do problema a que se propõe

Caracterizar um sistema como federado implica em considerar que este seja
constituído por sistemas menores, controlados por entidades diferentes, que se
comunicam através de um protocolo arbitrário [ref]. Esta arquitetura faz com que 
nenhuma entidade individual exerça controle absoluto sobro as informações da 
federação, visto que cada um dos sistemas possui sua próprias políticas, 
infraestrutura, usuários, e conteúdo.

% ref to:
% http://softwarelivre.org/noosfero/blog/o-que-sao-redes-sociais-virtuais-federadas

Um exemplo recorrente de federação é o serviço de e-mails da Internet [ref]. Além
dos servidores de e-mail consolidados, qualquer entidade pode configurar um novo, e 
cada um deles vai contar com seus próprios clientes, e serão mantidos e gerenciados
de maneiras diferentes. Ainda assim, compartilham um protocolo de comunicação (SMTP,
por exemplo) para que a troca de mensagens entre clientes de diferentes servidores
seja possível.

O projeto de um sistema que suporte federação é mais simples se houver interesse
prévio em concordar com alguma especificação existente. Por outro lado, evoluir um
sistema 


O suporte a um protocolo de comunicação qualquer deve ser previsto no projeto de
qualquer sistema, o que depende de um interesse prévio em suportar a federação. Uma
segunda alternativa 

% gancho: por que motivo é interessante ser interoperável?

% (portais de quarta geração...)...
% the five touchpoints: Gen-4 Portal Functionality: From Unification to Federation

Existe um conjunto de atividades envolvidas nos problemas que  a integração
de sistemas.

% ref?
\begin{itemize}
  \item{Versatilidade na descoberta e compartilhamento da identidade dos usuários
        de diferentes redes, o que inclui a composição de um perfil público e
        compartilhado. Um usuário deve ser capaz de utilizar uma rede federada com
        um único perfil}
  \item{Compartilhamento do conteúdo e das atividades dos usuários entre redes,
        proporcionando mecanismos de interação. Criações e interações devem ser
        retransmitidas mesmo que as redes integradas não solicitem explicitamente}
  \item{Consistência no estado do conteúdo compartilhado entre as redes, respeitando
        as políticas de privacidade de cada uma delas. Toda modificação em qualquer
        conteúdo deve ser espelhada para todas as redes envolvidas}
\end{itemize}

% Awareness?


\subsection{Redes sociais federadas}

% introduzir mídias sociais

Uma das motivações de sistemas federados trata da descentralização de informações,
ao mesmo tempo em que são oferecidos mecanismos necessários para a integração de
diversas fontes de conteúdo. Em consequência disto, a federação de sistemas provoca
impacto sobre a liberdade no controle de informações e privacidade dos usuários.

Enquanto estes argumentos não fazem sentido em uma rede social popular, em que os
serviços sejam controlados por uma única entidade, são significativos para redes que
podem ser mantidas por entidades independentes, o que engloba aplicações de
\textit{software} livre. Considerando que qualquer rede social livre pode contar com
diversos números de instâncias mantidas por entidades distintas, os benefícios da
federação são de grande interesse da comunidade.

\subsection{Compartilhamento de recursos}

\subsection{Gerenciamento de identidade}

\subsection{Consistência de informações}



\section{OBJETIVOS}
% ?
