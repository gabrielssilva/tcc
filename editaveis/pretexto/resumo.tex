\begin{resumo}[Resumo]
  A federação de redes sociais propõe a integração de usuários através de uma
  estrutura descentralizada, possibilitando a interoperabilidade entre serviços
  distintos de maneira transparente usando de protocolos de comunicação. Apesar da
  iniciativa da comunidade em federar redes sociais livres, não existe adoção a uma
  especificação padrão, o que dificulta o surgimento de sistemas federados. De forma
  a entender as dificuldades envolvidas na construção e padronização de serviços
  federados, foi realizado um estudo a respeito das especificações e implementações
  de interoperabilidade existentes no cenário de redes sociais livres. Em segundo
  lugar, foi desenvolvida uma prova de conceito no caso específico da plataforma
  Noosfero, onde foi implementada uma parte do protocolo Diaspora para a federação
  de usuários e conteúdos públicos, o que inclui a utilização de especificações
  intermediárias, como o WebFinger e o Salmon. Como resultado, foi possível
  implementar parte do protocolo Diaspora no Noosfero, evidenciando as medidas a
  serem tomadas na continuidade do desenvolvimento.

  \vspace{\onelineskip}

  \noindent
  \textbf{Palavras-chaves}: federação. redes sociais. Noosfero. Diaspora.
\end{resumo}
