\begin{resumo}[Abstract]
  \begin{otherlanguage*}{english}
    The federation of social networks aims at integrating users by means of a
    decentralized structure, enabling the interoperability among multiple social
    networks in a transparent way. Despite a few isolated initiatives in federating
    open social networks, there is no adoption of any standard, which hinders the
    emergence of new, effective federated systems.  To understand the difficulties
    in the development and standardization of federated services, we have conducted
    research on existing specifications and implementations of interoperability
    among social networks.  We have developed a federation proof of concept within
    the Noosfero platform, implementing a subset of the Diaspora protocol to
    federate users and public content, in addition to complementary specifications,
    such as Salmon and WebFinger. In this work, we introduce our results to
    federate Noosfero with Diaspora networks, pointing the required steps before
    further development. We aim to implement the Diaspora protocol within Noosfero,
    finishing its specification and improving its documentation, encouraging more
    projects to adopt this protocol.

    \vspace{\onelineskip}

    \noindent
    \textbf{Key-words}: federation. social networks. Noosfero. Diaspora.
  \end{otherlanguage*}
\end{resumo}
