\begin{resumo}[Abstract]
  \begin{otherlanguage*}{english}
    The federation of social networks proposes to integrate users through a
    decentralized structure, making the interoperability between distinct services
    possible in a transparent way by using communication protocols. Despite the
    initiative in the community to federate open social networks, there is no
    adoption of any standard, which hinders the emergence of new federated systems.
    In order to understand the difficulties in the development and standardization
    of federated services, it was conducted a research regarding the existing
    specifications and implementations of interoperability between social networks.
    Moreover, it was developed a proof of concept of federation within the Noosfero
    platform, implementing a subset of the Diaspora protocol to federate users and
    public content, in addition to complementary specifications, such as Salmon and
    WebFinger. As results, it was possible to federate Noosfero with Diaspora
    networks, pointing the steps to be taken before further development.
    
    \vspace{\onelineskip}
    
    \noindent 
    \textbf{Key-words}: federation. social networks. Noosfero. Diaspora.
  \end{otherlanguage*}
\end{resumo}
