\part{SISTEMAS FEDERADOS}

\chapter{ARQUITETURAS DE REDES DE COMUNICAÇÃO}

% Tipos de redes de comunicação...

Redes de comunicação podem assumir diferentes arquiteturas, de acordo com a 
estruturação de seus nós. De acordo com (ref to Baran) redes podem assumir
configurações centralizadas, distribuídas, e descentralizadas. 

[--- imagem das três categorias de rede segundo Baran ---]

Outros autores propõem definições diferentes para redes distribuídas e
descentralizadas. (ref to Outro) usa o termo descentralização para compor os
conceitos de nós [--- descrever os conceitos aqui ---]. 

% Apresentar mais conceitos arquiteturais de redes.

\section{REDES DE COMUNICAÇÃO DESCENTRALIZADAS}

Por mais que existam diferentes definições para a arquitetura descentralizada de
redes de comunicação, o conceito converge para uma organização de nós que, apesar
de respeitar a noção de cliente-servidor, também é formada por um conjunto de
servidores que se comunicam de maneira distribuída. A Internet, por exemplo, surgiu
como um rede descentralizada onde um conjunto de clientes acessa, através de um
servidor local, uma rede maior de servidores distribuídos.

Apesar da Internet ser, por natureza, uma rede descentralizada, a popularização de 
grandes serviços e plataformas integradoras tende à adoção de uma estrutura  cada
vez mais centralizada. % Encontrar números que comprovam essa afirmação 

% benefícios de redes centralizadas
% centralização e descentralização da internet
% tendência para a centralização, apresentar os problemas de privacidade
% por que descentralizar novamente? (fazer um paralelo com serviços)

\section{A ADOÇÃO DE SISTEMAS FEDERADOS}

% Definição de federação, contextualização do problema a que se propõe
% (portais de quarta geração...)...


\chapter{PLATAFORMAS E PROTOCOLOS DE FEDERAÇÃO}

% apresentar a federação no contexto de software livre. Federação entre 
% instâncias diferentes, descentralização e agregação de conteúdo...

% Comentar a respeito dos vários sistemas livres federados.
% Diaspora*, Friendica, Hubzilla, GnuSocial...

% Descrever os protocolos existentes. Tratar do que abordam, 
% "padrões de interoperabilidade"

\section{OStatus}

O OStatus é uma especificação baseada em um conjunto de protocolos construída com a
intenção de oferecer uma estrutura completa para a padronização e interação entre 
redes sociais distintas. Foi inicialmente proposto por Evan Prodromou e implementado
no StatusNet, que posteriormente deu origem ao projeto Gnu social.

Apesar de Atom e RSS constituírem padrões corriqueiros para a implementação de feeds
na internet, tratam-se de especificações que não são capazes de permitir interações
em tempo real. A ideia do OStatus é combinar \textit{feeds} Atom com uma série de
mecanismos capazes de permitir identidade, confiabilidade na comunicação entre os
servidores.

O projeto conta com uma das maiores iniciativas de padronização de protocolos de
federação, tornando-se alvo de um grupo de trabalho da W3C desde 2012. Apesar de não
ter apresentado maiores avanços nos últimos anos, ainda há interesse da comunidade
em mantê-lo e utilizá-lo. 

\subsection{PubSubHubbub}

De forma a possibilitar as interações em tempo real previstas pelo  OStatus, é 
necessário utilizar algum tipo de mecanismo que, em combinação com o Atom, possa
entregar notificações 

O PubHubSub é um sistema de publicações de publicação e assinatura distribuído.
Possibilita que servidores se inscrevam em \textit{feeds}, e sejam notificados assim
que alguma alteração aconteça. A ideia é que uma série de serviços se inscrevem em
diretórios centrais (ou \textit{hubs}), expressando interesse em receber
notificações de atualização. Serviços inscritos devem identificar o tópico desejado
através de URLs, e oferecer um servidor disponível pela internet para que a
notificação possa ser realizada. Entidades interessadas em publicar algum tipo de
conteúdo incluem uma referência ao \textit{hub}. É de responsabilidade dos
produtores de conteúdo notificar o \textit{hub} em novas publicações, que por sua
vez são responsáveis por divulgar para cada um dos indivíduos inscritos.

Sem entrar em maiores detalhes a respeito da especificação, a comunicação dos
\textit{hubs} com as entidades que publicam e consomem os conteúdos acontecem sobre
HTTP, e identificadas por endereços URL. As notificações são baseadas na execução
de trechos arbitrários de código ativados a partir de \textit{endpoints} URL, 
seguindo a ideia de \textit{webhooks}.

\subsection{WebFinger}

% referenciar a RFC 7033

O WebFinger é um protocolo de descoberta de identidade que soluciona o problema do
compartilhamento das informações de usuários entre servidores remotos. A proposta é
que a partir de um atributo identificador de usuário e do endereço do seu servidor
de origem, seja possível garantir a existência do usuário e obter suas informações
públicas.

A especificação do WebFinger propõe que todos os recursos possam ser identificados
por uma URL, e que todas as solicitações e respostas sejam realizadas através de
requisições HTTP. Servidores que forneçam informações através do WebFinger devem
responder aos \textit{endpoints} definidos no protocolo com objetos JSON. Servidores
que pretendam consumir tais informações precisam conhecer apenas a URL do recurso de
interesse, o que pode ser encontrado a partir do identificador do usuário e do
domínio do servidor de origem (em um processo que pode ser classificado como LRDD). 

% referenciar LRDD: https://tools.ietf.org/html/draft-hammer-discovery-06 

No caso específico do OStatus, o WebFinger é utilizado no início de cada interação
entre usuários. Por exemplo, considerando um cenário em que um usuário pretende se 
inscrever no \textit{feed} de um usuário remoto, conhecendo seu identificador e o 
servidor de origem. Antes de solicitar a inscrição através de um servidor PubHubSub
e notificar o usuário remoto, o usuário local deve primeiro obter as informações
necessárias (como por exemplo a URL do \textit{feed} público) através de uma
requisição  WebFinger.

\subsection{Activity Streams}

O Activity Streams é uma especificação que propõe a padronização da representação
de atividades em redes sociais que tenham interesse em oferecer mecanismos de
integração. O emprego deste protocolo é justificado pela necessidade em registrar
as atividades do usuário no interior da rede, bem como pela representação através
de um formato que possa ser consumido por qualquer serviço interessado e autorizado.

A especificação mais atual do Activity Streams exige uma mensagem em formato JSON, 
incluindo uma série de propriedades que identificam a ação, o usuário responsável, 
e a entidade alvo. Considera-se ainda o envio de atividades complexas que contenham
referências a outras atividades ou entidades do sistema. 

Além da especificação que adota o padrão JSON, outras especificações utilizam o 
padrão Atom, com restrições semelhantes em relação ao conteúdo e formato das 
mensagens. O OStatus adota o protocolo especificado em Atom para o envio de entradas
de Activity Streams.

\subsection{Salmon}

Combinar soluções como Atom e PubHubSub permitem que conteúdos sejam publicados e
atualizados em tempo real. No entanto, a partir do ponto em que um conteúdo pode
ser consumido a partir de um número arbitrário de serviços, deve-se investir
esforço para garantir que o estado da discussão (como comentários e avaliações
relacionados a uma publicação) seja o mesmo em toda a rede. Mais especificamente,
qualquer entidade consumidora que atualize o estado da discussão deve notificar o
servidor fonte, que de alguma forma deve garantir um estado uniforme para o restante
da rede.

O Salmon é capaz de garantir a unificação da discussão entre a rede em que um
conteúdo foi publicado, e todos os serviços consumidores. Trata-se de uma 
especificação de troca de mensagens que define como as entidades consumidoras devem 
notificar os servidores fonte a respeito da atualização do estado de discussões, e 
como deve estes devem reagir a tais eventos.

Servidores interessados em implementar o protocolo devem incluir um 
\textit{endpoint} Salmon em seus \textit{feeds}, que será lido pelas entidades
consumidoras, e para o qual as mensagens devem ser enviadas sobre HTTP. Uma mensagem
Salmon se trata de uma entrada adicional no \textit{feed}, codificada em 
\textit{base64}, envolta por outra estrutura XML assinada digitalmente. Ao Receber a
mensagem, o servidor possui a liberdade de proceder conforme suas próprias 
políticas, desde que retorne uma resposta HTTP válida.

\section{Diaspora}

O projeto Diaspora surgiu com a intenção de implementar o conceito de redes sociais
descentralizadas em resposta aos problemas de liberdade e privacidade encontrados
em plataformas sociais privadas. Sua primeira versão foi lançada em Setembro de 2010
como fruto de uma campanha de financiamento coletivo, passando a ser completamente
governado pela comunidade a partir de Agosto de 2012.

% Descrever melhor o que é o Diaspora
% Baseado em OStatus?

A proposta dos desenvolvedores é evitar a centralização de conteúdo, portanto a 
falta de controle sobre as informações, construindo uma rede de instâncias pessoais
da plataforma, ou \textit{pods}. Cada servidor Diaspora reúne as informações dos 
seus usuários, mas em cooperação com outros servidores, possibilita a interação
e compartilhamento de informações entre usuários de diferentes \textit{pods}.

Além da própria plataforma, desenvolvida em Ruby on Rails, uma biblioteca que
implementa o protocolo de federação do Diaspora também está disponível no formato
de uma \textit{gem} Ruby. Todas as soluções são distribuídas sob a licença AGPL 
versão 3. 

O protocolo de federação do Diaspora foi definido para convergir com o OStatus
assim que o último passe a suportar o conceito de privacidade limitada. A
implementação leva em consideração a troca de mensagens em formato XML, respeitando
alguns conceitos básicos como a existência de usuários remotos e atualização remota
e retransmissão.

\subsection{Usuários Remotos}

Um conceito fundamental para a implementação de redes federadas é considerar a
existência de usuários remotos. A maioria das aplicações só possui o conceito de
usuários locais, que estão diretamente autenticados no serviço e possuem todas as
informações na base de dados local. No entanto, ao possibilitar a interação com
usuários de outras redes, usuários externos ao sistema precisam ser explicitamente
considerados na implementação das funcionalidades.

O Diaspora conceitualiza seus usuários em locais e externos. Enquanto usuários
locais respeitam a definição tradicional, os usuários externos interagem com a
aplicação através dos mecanismos de federação. É importante prever a existência de
usuários externos na modelagem de sistemas. Por este motivo, implementar federação
em aplicações já consolidadas pode exigir um certo esforço de refatoração. % ref?

\subsubsection{Capacidade de Retransmissão}

A retransmissão é essencial em sistemas federados, visto que interações em uma rede
eventualmente devem afetar \textit{pods} relacionados. A restrição implementada
pelo Diaspora indica que todas as notificações neste contexto sejam entregues tanto
aos usuários locais quanto aos usuários remotos. Adicionalmente, a notificação de
usuários locais não deve depender da resposta dos demais \textit{pods}.

Em configurações de integração mais complexas, a capacidade de retransmissão passa a
ser um requisito essencial para a troca de mensagens. Conside uma situação
hipotética em que \textit{pods} \textbf{A} e \textbf{B} são federados com o
\textit{pod} \textbf{C}, mas não entre si. Qualquer modificação em um conteúdo de 
\textbf{C} compartilhado com \textbf{A} e \textbf{B} deve afetar os três 
\textit{pods}. No entanto, se a modificação partir de \textbf{A}, há uma dificuldade
em notificar \textbf{B}, visto que o \textit{pod} em questão só reconhece a 
existência de \textbf{C}. A solução defendida pela implementação do Diaspora é que
\textbf{C} retransmita a notificação para todos os \textit{pods} com os quais o
conteúdo seja compartilhado. Isso garante que todos os sistemas federados envolvidos
em uma interação sejam notificados, contribuindo com consistência das informações.

\subsubsection{Troca de Mensagens}

O Diaspora define um conjunto de mensagens que delimitam as possíveis interações 
entre \textit{pods}.

\begin{itemize}
  \item{Compartilhamento de informações}
  \item{Publicações de conteúdo}
  \item{Comentários e reações a publicações}
  \item{Mensagens privadas}
\end{itemize}

A troca de mensagens segue a definição do protocolo Diaspora, que utiliza um
subconjunto do protocolo Salmon. De modo geral, restringe como a mensagem deve ser
construída e enviada para o \textit{endpoint} Salmon do \textit{pod} de destino. 

\subsection{Demais Iniciativas}
