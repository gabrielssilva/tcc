\part{REFERENCIAL TEÓRICO}

\chapter{REDES DESCENTRALIZADAS}

% Tipos de redes de comunicação...

\chapter{SISTEMAS FEDERADOS}

% Definição de federação, contextualização do problema a que se propõe
% (portais de quarta geração...)...

\section{PLATAFORMAS E PROTOCOLOS DE FEDERAÇÃO}

% apresentar a federação no contexto de software livre. Federação entre 
% instâncias diferentes, descentralização e agregação de conteúdo...

% Comentar a respeito dos vários sistemas livres federados.
% Diaspora*, Friendica, Hubzilla, GnuSocial...

% Descrever os protocolos existentes. Tratar do que abordam, 

\subsection{Salmon}

\subsection{Webfinger}

\subsection{OStatus}

\subsection{Diaspora}

O projeto Diaspora surgiu com a intenção de implementar o conceito de redes sociais
descentralizadas em resposta aos problemas de liberdade e privacidade encontrados
em plataformas sociais privadas. Sua primeira versão foi lançada em Setembro de 2010
como fruto de uma campanha de financiamento coletivo, passando a ser completamente
governado pela comunidade a partir de Agosto de 2012.

A proposta dos desenvolvedores é evitar a centralização de conteúdo, portanto a 
falta de controle sobre as informações, construindo uma rede de instâncias pessoais
da plataforma, ou \textit{pods}. Cada servidor Diaspora reúne as informações dos 
seus usuários, mas em cooperação com outros servidores, possibilita a interação
e compartilhamento de informações entre usuários de diferentes \textit{pods}.

Além da própria plataforma, desenvolvida em Ruby on Rails, uma biblioteca que
implementa o protocolo de federação do Diaspora também está disponível no formato
de uma \textit{gem} Ruby. Todas as soluções são distribuídas sob a licença AGPL 
versão 3. 

O protocolo de federação do Diaspora foi definido para convergir com o OStatus
assim que o último passe a suportar o conceito de privacidade limitada. A
implementação leva em consideração a troca de mensagens em formato XML, respeitando
alguns conceitos básicos como a existência de usuários remotos e atualização remota
e retransmissão.

\subsubsection{Usuários Remotos}

Um conceito fundamental para a implementação de redes federadas é considerar a
existência de usuários remotos. A maioria das aplicações só possui o conceito de
usuários locais, que estão diretamente autenticados no serviço e possuem todas as
informações na base de dados local. No entanto, ao possibilitar a interação com
usuários de outras redes, usuários externos ao sistema precisam ser explicitamente
considerados na implementação das funcionalidades.

O Diaspora conceitualiza seus usuários em locais e externos. Enquanto usuários
locais respeitam a definição tradicional, os usuários externos interagem com a
aplicação através dos mecanismos de federação. É importante prever a existência de
usuários externos na modelagem de sistemas. Por este motivo, implementar federação
em aplicações já consolidadas pode exigir um certo esforço de refatoração. % ref?

\subsubsection{Capacidade de Retransmissão}

A retransmissão é essencial em sistemas federados, visto que interações em uma rede
eventualmente devem afetar \textit{pods} relacionados. A restrição implementada
pelo Diaspora indica que todas as notificações neste contexto sejam entregues tanto
aos usuários locais quanto aos usuários remotos. Adicionalmente, a notificação de
usuários locais não deve depender da resposta dos demais \textit{pods}.

Em configurações de integração mais complexas, a capacidade de retransmissão passa a
ser um requisito essencial para a troca de mensagens. Conside uma situação
hipotética em que \textit{pods} \textbf{A} e \textbf{B} são federados com o
\textit{pod} \textbf{C}, mas não entre si. Qualquer modificação em um conteúdo de 
\textbf{C} compartilhado com \textbf{A} e \textbf{B} deve afetar os três 
\texit{pods}. No entanto, se a modificação partir de \textbf{A}, há uma dificuldade
em notificar \textbf{B}, visto que o \textit{pod} em questão só reconhece a 
existência de \textbf{C}. A solução defendida pela implementação do Diaspora é que
\textbf{C} retransmita a notificação para todos os \textit{pods} com os quais o
conteúdo seja compartilhado. Isso garante que todos os sistemas federados envolvidos
em uma interação sejam notificados, contribuindo com consistência das informações.

\subsubsection{Troca de Mensagens}

O Diaspora define um conjunto de mensagens que delimitam as possíveis interações 
entre \textit{pods}.

\begin{itemize}
  \item{Compartilhamento de informações}
  \item{Publicações de conteúdo}
  \item{Comentários e reações a publicações}
  \item{Mensagens privadas}
\end{itemize}

A troca de mensagens segue a definição do protocolo Diaspora, que utiliza um
subconjunto do protocolo Salmon. De modo geral, restringe como a mensagem deve ser
construída e enviada para o \textit{endpoint} Salmon do \textit{pod} de destino. 

\subsection{Demais Iniciativas}
