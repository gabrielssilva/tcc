\part{SISTEMAS FEDERADOS}

\chapter{ARQUITETURAS DE REDES DE COMUNICAÇÃO}

% Tipos de redes de comunicação...

Redes de comunicação podem assumir diferentes arquiteturas, de acordo com a 
estruturação de seus nós. De acordo com (ref to Baran) redes podem assumir
configurações centralizadas, distribuídas, e descentralizadas. 

[--- imagem das três categorias de rede segundo Baran ---]

Outros autores propõem definições diferentes para redes distribuídas e
descentralizadas. (ref to Outro) usa o termo descentralização para compor os
conceitos de nós [--- descrever os conceitos aqui ---]. 

% Apresentar mais conceitos arquiteturais de redes.

\section{REDES DE COMUNICAÇÃO DESCENTRALIZADAS}

Por mais que existam diferentes definições para a arquitetura descentralizada de
redes de comunicação, o conceito converge para uma organização de nós que, apesar
de respeitar a noção de cliente-servidor, também é formada por um conjunto de
servidores que se comunicam de maneira distribuída. A Internet, por exemplo, surgiu
como um rede descentralizada onde um conjunto de clientes acessa, através de um
servidor local, uma rede maior de servidores distribuídos.

Apesar da Internet ser, por natureza, uma rede descentralizada, a popularização de 
grandes serviços e plataformas integradoras tende à adoção de uma estrutura  cada
vez mais centralizada. % Encontrar números que comprovam essa afirmação 

% benefícios de redes centralizadas
% centralização e descentralização da internet
% tendência para a centralização, apresentar os problemas de privacidade
% por que descentralizar novamente? (fazer um paralelo com serviços)

\section{A ADOÇÃO DE SISTEMAS FEDERADOS}

% Definição de federação, contextualização do problema a que se propõe
% (portais de quarta geração...)...


\chapter{PLATAFORMAS E PROTOCOLOS DE FEDERAÇÃO}

% apresentar a federação no contexto de software livre. Federação entre 
% instâncias diferentes, descentralização e agregação de conteúdo...

% Comentar a respeito dos vários sistemas livres federados.
% Diaspora*, Friendica, Hubzilla, GnuSocial...

% Descrever os protocolos existentes. Tratar do que abordam, 
% "padrões de interoperabilidade"

\section{OStatus}

\subsection{Salmon}

\subsection{Webfinger}

\section{Diaspora}

O projeto Diaspora surgiu com a intenção de implementar o conceito de redes sociais
descentralizadas em resposta aos problemas de liberdade e privacidade encontrados
em plataformas sociais privadas. Sua primeira versão foi lançada em Setembro de 2010
como fruto de uma campanha de financiamento coletivo, passando a ser completamente
governado pela comunidade a partir de Agosto de 2012.

% Descrever melhor o que é o Diaspora
% Baseado em OStatus?

A proposta dos desenvolvedores é evitar a centralização de conteúdo, portanto a 
falta de controle sobre as informações, construindo uma rede de instâncias pessoais
da plataforma, ou \textit{pods}. Cada servidor Diaspora reúne as informações dos 
seus usuários, mas em cooperação com outros servidores, possibilita a interação
e compartilhamento de informações entre usuários de diferentes \textit{pods}.

Além da própria plataforma, desenvolvida em Ruby on Rails, uma biblioteca que
implementa o protocolo de federação do Diaspora também está disponível no formato
de uma \textit{gem} Ruby. Todas as soluções são distribuídas sob a licença AGPL 
versão 3. 

O protocolo de federação do Diaspora foi definido para convergir com o OStatus
assim que o último passe a suportar o conceito de privacidade limitada. A
implementação leva em consideração a troca de mensagens em formato XML, respeitando
alguns conceitos básicos como a existência de usuários remotos e atualização remota
e retransmissão.

\subsection{Usuários Remotos}

Um conceito fundamental para a implementação de redes federadas é considerar a
existência de usuários remotos. A maioria das aplicações só possui o conceito de
usuários locais, que estão diretamente autenticados no serviço e possuem todas as
informações na base de dados local. No entanto, ao possibilitar a interação com
usuários de outras redes, usuários externos ao sistema precisam ser explicitamente
considerados na implementação das funcionalidades.

O Diaspora conceitualiza seus usuários em locais e externos. Enquanto usuários
locais respeitam a definição tradicional, os usuários externos interagem com a
aplicação através dos mecanismos de federação. É importante prever a existência de
usuários externos na modelagem de sistemas. Por este motivo, implementar federação
em aplicações já consolidadas pode exigir um certo esforço de refatoração. % ref?

\subsubsection{Capacidade de Retransmissão}

A retransmissão é essencial em sistemas federados, visto que interações em uma rede
eventualmente devem afetar \textit{pods} relacionados. A restrição implementada
pelo Diaspora indica que todas as notificações neste contexto sejam entregues tanto
aos usuários locais quanto aos usuários remotos. Adicionalmente, a notificação de
usuários locais não deve depender da resposta dos demais \textit{pods}.

Em configurações de integração mais complexas, a capacidade de retransmissão passa a
ser um requisito essencial para a troca de mensagens. Conside uma situação
hipotética em que \textit{pods} \textbf{A} e \textbf{B} são federados com o
\textit{pod} \textbf{C}, mas não entre si. Qualquer modificação em um conteúdo de 
\textbf{C} compartilhado com \textbf{A} e \textbf{B} deve afetar os três 
\texit{pods}. No entanto, se a modificação partir de \textbf{A}, há uma dificuldade
em notificar \textbf{B}, visto que o \textit{pod} em questão só reconhece a 
existência de \textbf{C}. A solução defendida pela implementação do Diaspora é que
\textbf{C} retransmita a notificação para todos os \textit{pods} com os quais o
conteúdo seja compartilhado. Isso garante que todos os sistemas federados envolvidos
em uma interação sejam notificados, contribuindo com consistência das informações.

\subsubsection{Troca de Mensagens}

O Diaspora define um conjunto de mensagens que delimitam as possíveis interações 
entre \textit{pods}.

\begin{itemize}
  \item{Compartilhamento de informações}
  \item{Publicações de conteúdo}
  \item{Comentários e reações a publicações}
  \item{Mensagens privadas}
\end{itemize}

A troca de mensagens segue a definição do protocolo Diaspora, que utiliza um
subconjunto do protocolo Salmon. De modo geral, restringe como a mensagem deve ser
construída e enviada para o \textit{endpoint} Salmon do \textit{pod} de destino. 

\subsection{Demais Iniciativas}
