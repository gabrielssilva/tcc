\chapter{Conclusions}
\label{chapter:7}

Our report on the existing federation initiatives helped to identify
\textit{what are the protocols in use for federation of social networks and
what is the status of their development and adoption}. We described several
standards that are used to build decentralized social networks, where we can
highlight the Diaspora protocol, and OStatus and derived projects, such as
Mastodon. While evaluating the adoption of these technologies, we could verify
the lack of standardization, identifying a scenario where different projects
choose to follow distinct strategies to better suit individual needs.

The Diaspora protocol provides a reliable process for finding users and
exchanging messages among servers. By implementing it in Noosfero, we
were able to federate an instance with both Noosfero and Diaspora
servers, showing that it is feasible to implement federation in an
existing application.

It is important to note that the Diaspora protocol suits only a subset
of the existing projects, from what follows that it is most probably not
a ``silver bullet'' for building a larger federation.

Even though we can point projects that successfully implemented a
decentralized network, such as Hubzilla
\footnote{\url{https://github.com/redmatrix/hubzilla}},
which also supports the Diaspora protocol, these are usually designed
for a decentralized context and aim to provide federation out of the box.
An additional challenge lies with projects that were not designed as
such since the beginning but still want to support federation.

Our proof of concept helps to answer \textit{what engineering problems need to
be solved in order to add support for federation in a platform that was not
conceived with federation as a requirement?} To start with, our implementation
would require more effort if Noosfero features were not already somewhat
compatible with the specification. For example, Noosfero already supported
asymmetric relationships (``follows'') besides symmetric ones (``friends'').
If that was not the case, the implementation would turn harder than it was.

There are a couple points that we want to list regarding the implementation of
the Diaspora protocol in existing social networks. These are either challenges
that can also be faced or aspects that need attention when adding federation
support in projects that were not designed to be federated. These findings also
offer some insight on how the Diaspora protocol can be used, and we take the
opportunity to point some limitations.


\section{Lessons learned on supporting a federation protocol}

Beforehand, it is important to take into account that your data model will have
to be modified to contemplate Diaspora entities. For instance, all exchanged
entities must have a global identifier (GUID), and each user requires a pair of
RSA keys to properly send and read private Salmon messages.

The protocol may also affect the way the project handles users and contents,
mostly because all remote data must be stored in the local database, what
includes remote profiles and contents. It most likely means that the
application must foresee the existence of remote users and provide some level
of interaction, like publishing contents and sharing with the rest of the
network.

There are also some points that can help to exchange entities among servers.
When sending or receiving these entities, besides the message format, it is
important to identify the endpoints to be used for the HTTP requests. Remote
servers will use analogue endpoints to send its own entities when establishing
the communication, so it will be easier to start by choosing a small set of
entities and sending them from a remote server, implementing the local receiver
endpoints as needed. Contacts and status messages are a good choice of entities
to begin with.

Sending and Handling entities should be done in background, since it
involves network usage and opening and creating Salmon envelopes, which
can be computationally expensive. Whereas part of the work is to build
and exchange Salmon messages, one should also consider using some tool
that implements it, such as the diaspora federation Ruby library
\footnote{\url{https://github.com/diaspora/diaspora_federation}}.

The reliability of the message sending must be assured by each server. It may
be interesting to keep track of the statuses of each pod, for example storing
timestamps and number of tentatives of failed requests. Pods do not inform the
network of their status, so everything is still made based on the HTTP
responses status. Considering that the failure to deliver messages affects the
coherence of the information in the network, the protocol should be reviewed to
handle scenario formally.

It is also important to point that a server may implement several federation
protocols, i.e. servers A and C support two different federation protocols,
and server B wants to join both networks. This use case will affect the
implementation, since public interactions from users in B with A or C may be
potentially replicated.

Consider that a user from B interacts with both A and C servers, the
state of public contents will need to be the same across all networks.
In this case, A and C do not acknowledge each other, so the responsibility
to relay the modifications lies with B. Also because A and C are isolated,
the \textit{relayability} restriction is not described in any of the
specifications, and it is up to the local server to decide whether or not
the messages should be relayed.

The relaying problem is also present when all servers support the
same protocol, and it is somewhat covered by Diaspora's specification.
The idea is that relaying messages to pods that are in some way related
to the interaction, it will be easier to find references to new pods,
helping the servers to establish new connections. Since every federated
feature depends on a reference to the remote pod, it is important to
provide the means to broaden the network.


\section{Weaknesses and future work}

The federation tries to provide a seamless platform for interaction across
networks, and part of this feature is to use the same authentication
credentials for every server. For a user, it means to authenticate in one
server, access other server under any domain, and still be authorized
to share and interact. This feature is also in the scope of the federation
in Noosfero, and it may use the Diaspora protocol infrastructure and WebFinger
identities.

Regarding the support of the Diaspora protocol, contents with limited
visibility were not in the scope of this proof of concept, and it is an
important step to extend the federation features turning the network privacy
aware. This will require to match Noosfero circles with Diaspora aspects,
delivering the content only to some users.

For now, only plain text messages are being shared across servers. Since
Noosfero is also a CMS, it is our interest to federate rich contents,
such as articles with embed HTML and images. The protocol currently offers
the means to share static images, so it will probably be the case to extend
its custom entities.

It is also important to support direct messages, events and post reactions
(such as a ``Like''), showing some convergence with the Diaspora interaction
model, and increasing the use cases of the federated network.

Whereas we were able to implement the protocol and join a network of Diaspora
pods, it is clear the specification has to evolve in the ways it deals with
content replication and reliability of the messages delivery.

