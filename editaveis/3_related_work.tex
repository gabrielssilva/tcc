\chapter{Related Work}
\label{chapter:3}

Decentralized social networks are well established in early works.
Au Yeung et al. \cite{yeung2009} describes decentralized models for social
networks and proposes the use of technologies such as linked data. Chao et
al. \cite{chao2012} further explores this method describing seamless
interaction and single point of access to the network, important concepts
for federated systems.

In addition to reference models, several works propose implementations
for decentralized infrastructures. Buchegger et al. \cite{p2pInsights2009}
propose an infrastructure for P2P networks, explore issues related to
decentralized social networks, such as exchanging messages over a decentralized
architecture and encryption.  They also provide insights on asynchronous
message exchange and the use of a centralized storage mechanism to avoid
content replication.

There are other works that propose implementations of decentralized networks.
Boelmann et al. \cite{sonet2013} explore privacy and security issues. Aberer et
al. \cite{my32012} work on content replication, exploring the impacts of
availability and performance on the propagation of messages.

Bielenberg et al. \cite{diasporaGrowth2012} explore the growth of Diaspora
networks, but more importantly, they give an overview of privacy in the given
platform. The authors show that, at the time, users preferred to join popular
and reliable servers, indicating that they do not host their own data, even
though this is the proposal of decentralized networks.

Finally, Fan et al. \cite{snsapi2014} propose a different approach to develop a
decentralized social network, presenting a proof of concept with a large
network. The proposed approach is to, instead of building a decentralized
network, decentralize ordinary social networks by providing a layer to relay
contents from one network to another.

Even though there are some implementation proposals, we choose to explore the
approach of the Diaspora project. The discussion regarding the implementation
of federation protocols in existing projects is still incipient, and constitutes
a motivation to proceed with the proof of concept we propose.

Our goal is not to present another federation infrastructure, but to
investigate the implementation of existent protocols in projects that were not
designed to be federated, and differently from the related works, to report our
experiences implementing the said proof of concept.
