\chapter{ARQUITETURAS DE REDES DE COMUNICAÇÃO}

% Tipos de redes de comunicação...

Redes de comunicação podem assumir diferentes arquiteturas, de acordo com a 
estruturação de seus nós. De acordo com (ref to Baran) redes podem assumir
configurações centralizadas, distribuídas, e descentralizadas. 

[--- imagem das três categorias de rede segundo Baran ---]

Outros autores propõem definições diferentes para redes distribuídas e
descentralizadas. (ref to Outro) usa o termo descentralização para compor os
conceitos de nós [--- descrever os conceitos aqui ---]. 

% Apresentar mais conceitos arquiteturais de redes.

\section{REDES DE COMUNICAÇÃO DESCENTRALIZADAS}

Por mais que existam diferentes definições para a arquitetura descentralizada de
redes de comunicação, o conceito converge para uma organização de nós que, apesar
de respeitar a noção de cliente-servidor, também é formada por um conjunto de
servidores que se comunicam de maneira distribuída. A Internet, por exemplo, surgiu
como um rede descentralizada onde um conjunto de clientes acessa, através de um
servidor local, uma rede maior de servidores distribuídos.

Apesar da Internet ser, por natureza, uma rede descentralizada, a popularização de 
grandes serviços e plataformas integradoras tende à adoção de uma estrutura  cada
vez mais centralizada. % Encontrar números que comprovam essa afirmação 

% benefícios de redes centralizadas
% centralização e descentralização da internet
% tendência para a centralização, apresentar os problemas de privacidade
% por que descentralizar novamente? (fazer um paralelo com serviços)


\section{ARQUITETURA DE SISTEMAS FEDERADOS}

% como o conceito de redes federadas leva à arquitetura de sistemas federadas?

% Definição de federação, contextualização do problema a que se propõe
% (portais de quarta geração...)...

\subsection{FEDERAÇÃO DE SISTEMAS}

