\chapter{Implementação}

Com a contribuição deste trabalho, deve ser possível que um usuário do Noosfero
consiga encontrar e seguir a atividade de usuários de alguma instância do Diaspora.
Os usuários remotos descobertos devem possuir um perfil limitado no Diaspora, para
que as publicações na rede de origem sejam replicadas no Noosfero.

A implementação destas funcionalidades deve cobrir os pontos a seguir.

\begin{itemize}
  \item{Descoberta de usuários em um \textit{pod} do Diaspora.}
  \item{Resposta às consultas das informações do servidor Noosfero e da identidade
        ou informações do perfil de usuários locais.}
  \item{Envio de mensagens Salmon públicas e privadas para transportar as entidades
        que representam a interação social ou publicações.}
  \item{Recebimento de mensagens públicas que transportam entidades que representam
        novas publicações, criando publicações locais respectivas.}
\end{itemize}

A documentação da biblioteca utilizada para a implementação do protocolo não cobre
todos os aspectos da especificação, portanto foi necessário instanciar dois
\textit{pods} locais do Diaspora para analisar a comunicação. A comunicação do
Diaspora depende que cada um dos \textit{pods} responda a HTTPS e seja capaz de
resolver os nomes dos demais servidores.

Para os testes desse trabalho foram utilizadas duas máquinas virtuais com Debian 8
em rede privada, com os nomes registrados no arquivo de \textit{hosts} para a
resolução local. Em cada uma das máquinas, o Diaspora foi servido por um servidor
NGINX com SSL configurado com um certificado auto-assinado. Para que a comunicação
não fosse prejudicada pela verificação dos certificados, eles foram manualmente
adicionados aos certificados reconhecidos em cada uma das máquinas.

\section{AUTENTICAÇÃO COM O DIASPORA}

% TODO: inicialmente acreditava-se que seria necessário...

O Diaspora não disponibiliza nenhum \textit{endpoint} de \textit{login} em sua API,
mas implementa o papel de fornecedor de identidades OpenID. Portanto, a fim de
utilizar as credenciais do Diaspora, é necessário que o Noosfero possa desempenhar o
papel de cliente OpenID.

O Diaspora implementa apenas uma estratégia de autenticação através de perfis
locais. Mesmo que o Noosfero também fornecesse identidades OpenID, não seria
possível acessar uma instância do Diaspora com usuários do Noosfero, o que reforçou
a decisão de implementar apenas a função de cliente OpenID por enquanto.

% Visão geral do OpenID
% ref: http://openid.net/specs/openid-connect-core-1_0.html
O OpenID é um padrão de fornecimento e verificação de identidades construído sobre a
segunda versão do protocolo OAuth. De modo geral, permite que clientes verifiquem a
identidade de usuários através de um serviço disponibilizado por servidores de
autorização.

% OpenID Connect vs OAuth

No caso da federação com o Diaspora, é importante implementar o registro dinâmico em
novos provedores. Visto que podem existir diversos \textit{pods} Diaspora, e um
usuário pode usar qualquer um destes como fornecedor de identidade. O Noosfero deve
ser capaz de solicitar o registro dinamicamente em qualquer servidor OpenID a partir
do seu endereço.

\subsection{Desenvolvimento do plugin OpenID Client}

Apesar do Noosfero já oferecer suporte à autorização com OAuth 1.0 através de um
\textit{plugin}, foi necessário implementar o suporte ao OpenID. A decisão foi criar
um novo \textit{plugin} que transforme o Noosfero em um cliente OpenID.

A \textit{gem} openid\char`_connect foi utilizada na implementação do consumidor. A
biblioteca já oferece as diretrizes de descoberta, registro de clientes e
autenticação, todas interações que envolvem requisições HTTP ao servidor fornecedor.
Um perfil externo também é criado no Noosfero para armazenar algumas informações do
perfil remoto, como \textit{link} para o seu perfil, ou sua imagem do avatar.

Os passos realizados pelo \textit{plugin} na autenticação de um usuários são:

\begin{enumerate}
  \item{O usuário digita o endereço do fornecedor OpenID de sua escolha;}
  \item{O Noosfero tenta descobrir informações a respeito do provedor, solicitando
        informações do emissor de identidades;}
  \item{Caso a resposta do provedor seja válida, o Noosfero solicita o registro como
        um novo cliente, solicitando acesso a informações necessárias para a criação
        de um perfil externo;}
  \item{A requisição é redirecionada ao fornecedor OpenID, onde o usuário deve se
        autenticar, e revisar a solicitação enviada pelo Noosfero;}
  \item{Se o usuário se autenticar no seu provedor e aprovar as informações
        solicitadas, a resposta do servidor será usada na criação de um perfil
        externo, e o usuário será autenticado no Noosfero}
\end{enumerate}

\section{DESCOBERTA DE USUÁRIOS}

A especificação do Diaspora propõe a descoberta de usuários através do WebFinger
para a consulta de identidades, e do hCard para o compartilhamento das informações
do perfil. O protocolo ainda segue a implementação do WebFinger que responde em
formato XML, considerada legada.

A partir de um identificador no formato \textit{nome do usuário@servidor diaspora}, o
servidor remoto será consultado, e deve retornar uma resposta em formato XML a
respeito da identidade do usuário consultado. Após uma requisição WebFinger bem
sucedida, o servidor remoto é novamente consultado pelo hCard do usuário, por sua
vez em formato HTML, contendo as informações do perfil público.

No Noosfero, a descoberta de usuários remotos foi implementada a partir da busca de
pessoas. Se a \textit{string} de busca conter o símbolo "@", a string é quebrada
no formato \textit{nome@host}, e o processo descrito anteriormente é executado
para os atributos extraídos. Visto que trata-se de uma busca e os resultados são
imediatamente renderizados pelo servidor, por enquanto toda a operação é realizada
em \textit{foreground}.

Já que a implementação de descoberta é baseada no padrão WebFinger, apesar de seguir
o protocolo do Diaspora, é possível encontrar usuários em qualquer aplicação que o
implemente neste formato, não só em instâncias do Diaspora. A partir do momento em
que o Noosfero também responda a WebFinger e hCard nestes formatos, a implementação
do WebFinger que já existe pode ser substituída inclusive na descoberta de usuários
entre instâncias do Noosfero.

\section{Troca de Mensagens}

% troca de mensagens entre redes
% gem diaspora federation

% Requisições: (request -> response)


% Adicionar contato:
% - Quem adiciona:
% -- descobre receive url
% -- gera mensagem salmon
% -- send private (contact)
% - quem é adicionado:
% -- descobre o usuário de forma semelhante à busca


% Em uma nova publicação:
% - Quem escreve:
% -- post locally
% -- publish to hub AND sendpublic
% -- receive locally?
% - Quem recebe:
% -- receive locally

% e sobre retry?

% A implementar no Noosfero:
% -- mostrar perfil para usuários externos

% Salmon:
% - par de chaves para cada usuario do Noosfero
% - Gerar na criação OU quando tentar acessar

% Endpoints do Diaspora precisam estar no /, já que a gem pega do id@host
% deprecated pelo protocolo, mas implementação no diaspora ainda usa

% Todas as entidades do protocolo Diaspora precisam de um GUID/UUID

% Usuários externos possuem perfis e publicam scraps


\subsection{Recebimento de mensagens}

% diaspora\char`_federation?
% Extensível por plugins


\subsection{Envio de mensagens}

% DelayedJob + HTTP?
% Módulo de Federação em lib + action tracker
