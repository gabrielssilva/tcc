\chapter{Implementação}

Ao fim da implementação da federação com o Diaspora, deve ser possível que um
usuário do Noosfero consiga seguir usuários do Diaspora, organizando-os em grupos, e
visualizando as publicações deste usuário na sua rede de origem. Todas as interações
com estes conteúdos devem refletir impacto sobre todas as redes. Por exemplo, um
usuário do Noosfero deve ser capaz de comentar em uma das publicações provenientes
do Diaspora que apareçam no seu \textit{feed}, e esse comentário deve aparecer como
uma resposta ao autor no Diaspora.

% - Diagrama?

A implementação pode ser dividida em dois problemas. Em primeiro lugar, usuários
devem conseguir se autenticar no Noosfero com o perfil da rede remota. Em segundo
lugar, deve existir um mecanismo de troca de mensagens entre as instâncias do
Noosfero e os \textit{pods} do Diaspora para que a integração de conteúdos possa ser
realizada.


\section{Autenticação com o Diaspora}

O Diaspora não disponibiliza nenhum \textit{endpoint} de \texitt{login} em sua API,
mas implementa o papel de fornecedor de identidades OpenID. Portanto, a fim de
utilizar as credenciais do Diaspora, é necessário que o Noosfero possa desempenhar o
papel de cliente OpenID.

O Diaspora implementa apenas uma estratégia de autenticação através de perfis
locais. Mesmo que o Noosfero também fornecesse identidades OpenID, não seria
possível acessar uma instância do Diaspora com usuários do Noosfero, o que reforçou
a decisão de implementar apenas a função de cliente OpenID por enquanto.

% Visão geral do OpenID
% ref: http://openid.net/specs/openid-connect-core-1_0.html
O OpenID é um padrão de fornecimento e verificação de identidades construído sobre a
segunda versão do protocolo OAuth. De modo geral, permite que clientes verifiquem a
identidade de usuários através de um serviço disponibilizado por servidores de
autorização.

% OpenID Connect vs OAuth

No caso da federação com o Diaspora, é importante implementar o registro dinâmico em
novos provedores. Visto que podem existir diversos \textit{pods} Diaspora, e um
usuário pode usar qualquer um destes como fornecedor de identidade. O Noosfero deve
ser capaz de solicitar o registro dinamicamente em qualquer servidor OpenID a partir
do seu endereço.

\subsection{Desenvolvimento do plugin OpenID Client}

Apesar do Noosfero já oferecer suporte à autorização com OAuth 1.0 através de um
\textit{plugin}, foi necessário implementar o suporte ao OpenID. A decisão foi criar
um novo \textit{plugin} que transforme o Noosfero em um cliente OpenID.

A \textit{gem} openid\char\`_connect foi utilizada na implementação do consumidor. A
biblioteca já oferece as diretrizes de descoberta, registro de clientes e
autenticação, todas interações que envolvem requisições HTTP ao servidor fornecedor.
Um perfil externo também é criado no Noosfero para armazenar algumas informações do
perfil remoto, como \textit{link} para o seu perfil, ou sua imagem do avatar.

Os passos realizados pelo \textit{plugin} na autenticação de um usuários são:

\begin{enumerate}
  \item{O usuário digita o endereço do fornecedor OpenID de sua escolha;}
  \item{O Noosfero tenta descobrir informações a respeito do provedor, solicitando
        informações do emissor de identidades;}
  \item{Caso a resposta do provedor seja válida, o Noosfero solicita o registro como
        um novo cliente, solicitando acesso a informações necessárias para a criação
        de um perfil externo;}
  \item{A requisição é redirecionada ao fornecedor OpenID, onde o usuário deve se
        autenticar, e revisar a solicitação enviada pelo Noosfero;}
  \item{Se o usuário se autenticar no seu provedor e aprovar as informações
        solicitadas, a resposta do servidor será usada na criação de um perfil
        externo, e o usuário será autenticado no Noosfero}
\end{enumerate}

\section{Troca de Mensagens}

% troca de mensagens entre redes
% gem diaspora federation


\subsection{Recebimento de mensagens}

% diaspora\char`_federation?
% Extensível por plugins


\subsection{Envio de mensagens}

% DelayedJob + HTTP?
% Módulo de Federação em lib + action tracker
